\documentclass[a4paper,10pt]{article}

\usepackage{xltxtra} 
\usepackage{amsmath}
\usepackage[linkbordercolor={1 1 0}]{hyperref}
\usepackage[marginpar]{todo}

\usepackage[acronym,toc]{glossaries}

\usepackage{color}
\usepackage{xcolor}
\usepackage{listings}

\usepackage{subcaption}

\usepackage{caption}
\DeclareCaptionFont{white}{\color{white}}
\DeclareCaptionFormat{listing}{\colorbox{gray}{\parbox{\textwidth}{#1#2#3}}}
\captionsetup[lstlisting]{format=listing,labelfont=white,textfont=white}

\usepackage{tikz}
\usetikzlibrary{arrows,decorations.pathmorphing,backgrounds,positioning,fit,mindmap}

\setromanfont[Mapping=tex-text]{Linux Libertine O}
% \setsansfont[Mapping=tex-text]{DejaVu Sans}
% \setmonofont[Mapping=tex-text]{DejaVu Sans Mono}

%funny makros we want to use
\newcommand{\den}[1]{\ensuremath{[\![#1]\!]}}

%skill language definition
\lstdefinelanguage{skill}
{morekeywords={include,with,with,extends,annotation,const,auto,map,list,set,i1,i8,i16,i32,i64,v64,string,bool,f32,f64},
breakatwhitespace=true,
   breaklines=true,      
sensitive=false,
morecomment=[s]{/*}{*/},
morestring=[b]",
frameshape={nnn}{n}{y}{nyr},
}
\lstset{emph={%  
    tagged,class,indexed%
    },emphstyle={\color{red}\bfseries\underbar}%
}%

\title{The Serialization Killer Language}
\author{Timm Felden}
\date{\today}

\makeglossaries
%%%%% ACRONYMS %%%%%

\newacronym{abi}{ABI}{Application Binary Interface}
\newacronym{adt}{ADT}{Abstract Data Type}
\newacronym{api}{API}{Application Programming Interface}
\newacronym{skill}{SKilL}{Serialization Killer Language}
\newacronym{xml}{XML}{Extensible Markup Language}
\newacronym{xsd}{XSD}{XML Schema Definition Language}

%%%%% Actual Entries %%%%%

\newglossaryentry{builtInType}{
  name={built-in type},
  description={Any predefined type, that is not a compound type, i.e. annotations, booleans, integers, floats and strings.},
}
\newglossaryentry{userType}{
  name={user type},
  description={Any type, that is defined by the user using a type declaration},
}
\newglossaryentry{groundType}{
  name={ground type},
  description={Any type, that is not a compound type, i.e. the union of user defined types and built-in types},
}

\newglossaryentry{superType}{
  name={super type},
  description={If a user type A extends a type B, B is called the super type (of A).},
}
\newglossaryentry{subType}{
  name={sub type},
  description={If a user type A extends a type B, A is called the sub type (of B).},
}
\newglossaryentry{baseType}{
  name={base type},
  description={The root of a type tree, i.e. the farthest type reach able over the super type relation.},
}

\newglossaryentry{unknownType}{
  name={unknown type},
  description={We will call a type \textit{unknown}, if there is no \gls{visibleDeclaration} of the type. Such types must not occur in a declaration file, but they can be encountered in the serialization or deserialization process.},
}


\newglossaryentry{visibleDeclaration}{
  name={visible declaration},
  description={We will call a type declaration \textit{visible}, if it is defined in the local file, or in any file transitively reachable over include directives.},
}

\begin{document}
\maketitle

\begin{abstract}
  This paper presents an approach to serializing objects, which is tailored for usability, performance and portability. Unlike other general serialization mechanisms, we provide explicit support for extension points in the serialized data, in order to provide a maximum of upward compatibility and extensibility.
\end{abstract}

\renewcommand{\abstractname}{Acknowledgements}
\begin{abstract}
Main critics: Erhard Plödereder and Martin Wittiger.

Additional critics: Dominik Bruhn.
\end{abstract}

\todo{namensäquivalenz muss immernoch mit lowercase arbeiten, wie es u.a. in Ada ist; die Konvention ist CamelCase und backends sind angehalten CamelCase in die entsprechende konvention der ausgabesprache zu übertragen, z.B. indem man in ada alles in uppercase schreibt und vor den inneren großbuchstaben einen \_ einfügt, falls es nur einer ist}


\tableofcontents

% main parts of the document
\section{Motivation}

Many industrial and scientific projects suffer from platform or language dependent representation of their core data structures. These problems often cause software engineers to stick with outdated tools or even programming languages, thus causing a lot of frustration. This does not only increase the burden of hiring new project members, but can ultimately cause a project to die unnecessarily.

The approach presented in this paper provides means of platform and language independent specification of serializable data structures and therefore a safe way to let old tools of a tool suite talk to the knew ones, without even the need of recompiling the old ones. We set out to design a new language, because we believe, that the best language a programmer can use to write a new tool, is the language that he likes the most. We also hat the strict requirement to provide a solution that can describe an intermediate representation with stable parts that can be used for decades and unstable parts that may change on a daily basis, until it is know how the transported data has to be shaped.

In order to achieve this goal, we introduce two new concepts:

The first one is an easy to use specification language for data structures providing simple data types like integers and strings, abstract data types like sets and maps, type safe pointers, extension points and single inheritance. The specification language is modular allowing for more readable specifications.

The second one is a formalized mapping of specified types to a bitwise representation of stored objects. The mapping is very compact and therefore scalable, easy to understand and therefore easy to bind to a new language. It does encode the type system and can therefore provide a maximum of upward and downward compatibility, while maintaining type safety at the same time. It allows for a maximum of safety when it comes to manipulating data unknown to the generated interface, while maintaining high decoding and encoding speeds\footnote{The serialization and deserialization operations are linear in the size of the input/output file.}.

In contrast to other serialization formats such as \gls{xml}, the serializable data can not be viewed or modified with a text editor. This however does not mean, that it is not human readable, because one can provide a human usable editor to edit arbitrary \gls{skill} files.

An improvement over \gls{xml} is, that the reflective usage of stored data is expected to be quite rare, because the binding generator is able to generate an interface that ensures type safety of modifications and provides a nice integration into the target language. This leads to a situation, where it is possible to use files containing data of arbitrary types. If the data stored in the file is not used by a client, he does not have to pay for it with execution time or memory. It is also not required for a client to know the whole intermediate representation of a tool suite, but only the parts he is going to use in order to achieve his goals.

The expected file sizes range from 1 MiB to 2 GiB, while having virtually no relevant numerical limits in the file size\footnote{There are practical limits, such as Java having array lengths limited to ~$2^{31}$ or current file systems having a maximum file size limit that is roughly equivalent to the size of a file completely occupied by objects with a single field of a single byte. There will also be problems with raw I/O-Performance for very large files and an implementation of a binding generator, which can handle files not storable in the main memory is a tricky thing to do.}. Please note, that the skill file format is a lot more compact then equivalent \gls{xml} files would be.
It is expected, that files contain objects of hundreds of types with thousands of instances each. If a type in such a file would contain in average three pointers, the file size would still be around a mega byte, which is due to a very compact representation of stored data. This will also lead to high load and store performance, because the raw disk speed is expected to be the limiting factor.


\subsection{Scientific Contributions}

This section is a very concise representation of contributions, that in part have already been mentioned above and in parts, will be mentioned much later.

The suggested serialization format and serialization language offer all of the following features at the same time:
\begin{itemize}
 \item a small footprint and therefore high decoding speeds
 \item a fully reflective type encoding
 \item type safe storage of pointers both to known and unknown types\footnote{I.e. regular references and annotations.}
 \item the specification language is modular\footnote{I.e. it can be distributed over many files.} and easy to use
 \item no tool using a common intermediate representation has to know the complete specification. It is even possible to strip away or add individual fields of commonly used types.
 \item the coding is platform and language independent
 \item the coding offers a maximum of downward \textbf{and} upward compatibility
 \item a programmer is communicating through a generated interface, which allows programmers knowing nothing about skill to interact with it, ensures type safety easily. It also allows programmers to write tools in the language they know the best\footnote{This is a problem especially in the scientific community, where many researchers work on similar problems but on completely different tools.}
 \item stored data, that is never needed by a tool, will never be touched
\end{itemize}

Any of the arguments above have already been made in various contexts (e.g. \cite{ada05} §13.13, \cite{llvm}, \cite{xml11}, \cite{lamb87}), but there is, to the best of our knowledge, no solution bringing all these demands together into a single product that does the job automatically.


\subsection{Related Work}

There are many approaches similar to ours, but most of them have a different focus. This section shall provide a concise list of related approaches. For potential users of \gls{skill}, this might also present alternatives superior for individual use cases.

\subsection*{XML}

\gls{xml} is a file format (defined in \cite{xml11}). The main differences are:
\begin{itemize}
 \item[+] XML can be manipulated with a text editor\footnote{Whereas skill requires a special editor, which will be provided by us eventually.}.
 \item[+] It is easier to write a libXML for a new language than to write a \gls{skill} back-end\footnote{This is only a relevant point if no bindings exist for the language that you want to use.}.
 \item[-] XML is not an efficient encoding in terms of (disk-)space usage
 \item[-] XML is not type safe. This can be overcome partially by the \gls{xsd}.
 \item[-] XML does not provide references to other objects out of the box.
 \item[-] XML stores basically a tree, whereas a skill file contains an arbitrary amount of graphs of objects.
 \item[-] XML is usually accessed through a libXML, whereas \gls{skill} provides an API for each file format, thus a skill user does not require any \gls{skill} skills. To be fair, there are some language bindings, mainly for Java, which offer this benefit for XML as well.
\end{itemize}

\subsubsection*{XML Schema definitions}

The description language itself is more or less equivalent to most schema definition languages such as \gls{xsd} (as described in \cite{xsd11-1,xsd11-2}). The downside is that schema definitions have to operate on XML and can not directly be used with a binary format. There is also
no way to generate code for some client languages, including Ada, from a schema definition. For obvious reasons, there is no way to refer to other \gls{xml} documents from a \gls{skill} file. If this is a requirement, one might choose to stick with XML.

We feel that the \gls{skill} specification language is easier to use and existing specifications are a lot easier to read then \gls{xsd} files.

The type systems offered by \gls{skill} and \gls{xsd} are quite different, thus it might be worth a look which one better fits ones needs.

\subsubsection*{JAXP and xmlbeansxx}

For Java and C++, there are code generators, which can turn a XML schema file into code, which is able to deal with an XML in a similar way, as it is proposed by this work. In case of Java this is even in the standard library. The downside is, that, to our knowledge, this is only possible for Java and C++, thus it leaves us with portability issues. A minor problem of this approach is the lack of support for comment generation and the inefficient storage of serialized data.
An interesting observation is, that this approach deprives XML of its flexibility advantage over our solution.


\subsection*{ASN.1}

Is not powerful enough to fit our purpose.


\subsection*{IDL}

A concise description of IDL can be found in \cite{lamb87}. It seems not to be powerful enough and is certainly outdated. It is so old, that there are no bindings for any modern language. There is also not much documentation on further research on that area, thus creating a new approach with similar goals but modern techniques is in fact an option.

The published format is stated to be ASCII (\cite{lamb87} §2.4), which will cause similar efficiency problems as XML does, when large amounts of date are stored.


\subsection*{Apatche Thrift \& Protobuf}

Both lack sub-typing. Protobuf has an overly complex notation language. Both seem to be optimized for network protocols, thus they do not have storage pools, which are the foundation of our serialization approach and an absolute requirement for some of our features, such as hints (see section \ref{hints}).


\subsection*{Java Bytecode, LLVM/IR and others}

Although Java Bytecode (see \cite{jvmspec}) and the LLVM Intermediate Representation (see \cite{llvm}) are hand crafted formats, they served as a guiding example in many ways.


\subsection*{Language Specific}

Language specific is language specific and can therefore not be used to interface between subsystems written in different programming languages, without a lot of effort. Our aim is clearly a language independent and easy to use serialization format.
%hier sollte man eigentlich was zu ada §13.13 sagen; irgendwie passt das aber nicht
\nocite{ada05}


\subsection*{Language Interfaces}
Language Interfaces do not permit serialization capabilities. Most language only provide interfaces for C, with varying quality and varying degree of automation. A significant problem are interfaces between languages with different memory models.

Interfacing between languages with different type systems and memory models over a common C interface can be very inefficient.

\section{Syntax}

We use the tokens \verb/<id>/, \verb/<string>/, \verb/<int>/ and \verb/<comment>/. They equal C-style identifiers, strings, integer literals and comments respectively. We use a comment token, because we want to emit the comments in the generated code, in order to integrate nicely into the target languages documentation system.

\subsection{The Grammar}
The grammar of a \gls{skill} definition file is defined as:
\begin{verbatim}
UNIT :=
  INCLUDE*
  DECLARATION*

INCLUDE := 
  ("include"|"with") <string> ";"?

DECLARATION :=
  DESCRIPTION
  <id>
  ((":"|"with"|"extends") <id>)?
  "{" FIELD* "}"
  
FIELD :=
  DESCRIPTION
  (CONSTANT|DATA) ";"?
  
DESCRIPTION := 
  (RESTRICTION|HINT)*
  <comment>?
  (RESTRICTION|HINT)*
  
RESTRICTION :=
  "@" <id> ("(" (R_ARG ("," R_ARG)*)? ")")? ";"?
  
R_ARG := ("%"|<int>|<string>)

HINT := "!" <id> ";"?
  
CONSTANT :=
  "const" TYPE <id> "=" <int>
  
DATA :=
  "auto"? TYPE <id>
  
TYPE :=
  ("map" MAPTYPE
  |"set" SETTYPE
  |"list" LISTTYPE
  |ARRAYTYPE)
  
MAPTYPE :=
  "<" GROUNDTYPE ("," GROUNDTYPE)+ ">"
  
SETTYPE :=
  "<" GROUNDTYPE ">"
  
LISTTYPE :=
  "<" GROUNDTYPE ">"
  
ARRAYTYPE :=
  GROUNDTYPE
  ("[" (<id>|<int>)? "]")?
  
GROUNDTYPE :=
  (<id>|"annotation")

\end{verbatim}
Note: The Grammar is LL(1).\footnote{In fact it can be expressed as a single regular expression.}

Comment: The optional \texttt{;} at the end of includes or definitions are for convenience only.

\subsection{Reserved Words}

The language itself has only the reserved words \textbf{annotation}, \textbf{auto}, \textbf{const}, \textbf{include}, \textbf{with}, \textbf{bool}, \textbf{map}, \textbf{list} and \textbf{set}.

However, it is strongly advised against using any identifiers which form reserved words in a potential target language, such as Ada, C++, C\#, Java, JavaScript or Python. A complete list is given in appendix \ref{app:keywords}.

\subsection{Examples}

\begin{lstlisting}[label=blockExample,caption=Running Example,language=skill]
/** A source code location. */
SLoc {
  i16 line;
  i16 column;
  string path;
}

Block {
  SLoc begin;
  SLoc end;
  string image;
}

IfBlock : Block {
  Block thenBlock;
}

ITEBlock : IfBlock {
  Block elseBlock;
}
\end{lstlisting}

\subsubsection*{Includes, self references}

\begin{lstlisting}[label=example2a,caption=Example 2a,language=skill]
with "example2b.skill"

A {
  A a;
  B b;
}
\end{lstlisting}

\begin{lstlisting}[label=example2b,caption=Example 2b,language=skill]
with "example2a.skill"

B {
  A a;
}
\end{lstlisting}

\subsubsection*{Unicode}
The usage of non ASCII characters is completely legal, but discouraged.
\begin{lstlisting}[label=unicode,caption=Unicode Support,language=skill]
/** some arguably legal unicode characters. */
ö {
  ö ∀;
  ö €;
}
\end{lstlisting}


\section{Semantics}

This section will describe the meaning of individual keywords.

\subsection{Includes}
The file referenced by the with statement is processed as well. The declarations of all files reachable over \texttt{with} statements are collected, before any declaration is evaluated.

\subsection{\texttt{annotation}}
The type has a tag and a size, which allows it to be inserted at any annotation locations. This is useful in order to provide extension points in the file format. The file will still be readable by older implementations, which are not able to map any meaningful type into the annotation. A language binding is expected to provide something like an annotation proxy, which is used to represent annotation objects. If an application tries to get the object behind the proxy for an object of an unknown type, this will inevitably result in an error or exception. Therefore language bindings shall provide means of inspecting whether or not the type of the object behind an annotation is known.

As we will see in section \ref{serialization}, annotations are roughly equivalent to the type definition
\begin{verbatim}
annotation {
  v64 baseTypeName;
  v64 basePoolIndex;
}
\end{verbatim}
Of course, this is made transparent to the user and some language bindings will offer a special and type safe treatment of annotations.

An implementation may treat an annotation pointing to an object of unknown type like a null reference. This behavior is safe, because such an object can not exist in the serialized file, thus the annotation has not been updated upon removal of the complete type pool. This behavior might look rather strange at first glance but is an effect of lazy treatment of informations stored in skill files and completely safe.

\subsection{Sub Types}
A \gls{subType} of a \gls{userType} can be declared by appending the keyword \texttt{with} and the \gls{superType}s name to a declaration. In order to be well-formed, the sub type relation must remain acyclic and must not contain \glspl{unknownType}.

\subsection{\texttt{const}}
A const field can be used in order to create guards or version numbers, as well as overwriting deprecated fields with e.g. zeroes. The deserialization mechanism has to report an error if a constant field has an unexpected value.

\subsection{\texttt{auto}}
The language binding will create a field with the given type, but the content is transparent to the serialization mechanism. This is useful if the inference of the content of a field is likely to be faster then storing it, e.g. if it can be inferred lazily.

\subsection{Abstract Data Types}
\todo{rewrite section; it emerged from the fusion of two sections talking about ADTs}

\glspl{adt} showed to be useful and to increase the usability and understandability of the resulting code and file format.

\glspl{adt} are represented using arrays and pairs. \todo{ref encoding scheme which will be explained later; somewhat confusing}

The type system has a built-in notion of arrays, maps, lists and sets. Note that all of them are, from the view of serialization, equivalent to length encoded arrays. Their purpose is to increase the usability of the generated \gls{api}.


\subsection{Comments}
Comments provided in the skill file will be emitted into the generated code\footnote{If the target language does not allow for C-Style comments, the comments will be transformed in an appropriate way.}, thus allowing a user to get tool-tips in his IDE showing him this documentation. \todo{sprache!}



\section{The Type System}

\todo{ein paar einleitende worte}

\begin{figure}[h]
\centering
\tikz [small mindmap, every node/.style=concept, concept color=black!20,
grow cyclic,
level 1/.append style={level distance=4.2cm,sibling angle=65},
level 2/.append style={level distance=2.7cm,sibling angle=40},
level 3/.append style={level distance=2cm,sibling angle=35},
level 4/.append style={level distance=1.5cm,sibling angle=35}
]
\node [root concept]{All Types}[clockwise from=0] % root
child { node {User Types}}
child { node{Compound Types}[clockwise from=30]
  child{ node{map}}
  child{ node{set}}
  child{ node{list}}
  child{ node{array}}
}
child { node{Built-In Types}[clockwise from=-20]
  child{ node{string}}
  child{ node{Float}[clockwise from=0]
    child{ node{f64}}
    child{ node{f32}}
  }
  child{ node{Integer}[clockwise from=-30]
    child{ node{v64}}
    child{ node{i64}}
    child{ node{i32}}
    child{ node{i16}}
    child{ node{i8}}
  }
  child{ node{bool}}
  child{ node{annotation}}
};
\caption{Layout of the Type System}
\end{figure}

\Glspl{userType} can be seen as nonempty tuples over all types. \Glspl{builtInType} can be wrapped in order to give them special semantics. E.g. a time stamp can be created by:
\begin{lstlisting}[label=timeExample,caption=Time,language=skill]
time {
  /** seconds since 1.1.1970 0:00 UTC. */
  i64 date;
}
\end{lstlisting}

\subsection*{Common Abbreviations}

We will use some common abbreviations for sets of types in the rest of the manual:

Let \ldots
\begin{itemize}
 \item[\ldots] $\mathcal{T}$ be the set of all types.
 \item[\ldots] $\mathcal{U}$ be the set of all \glspl{userType}.
 \item[\ldots] $\mathcal{I}$ be the set of all integer types, i.e. $\{\texttt{i8},\texttt{i16},\texttt{i32},\texttt{i64},\texttt{v64}\}$.
 \item[\ldots] $\mathcal{B}$ be the set of all \glspl{builtInType}.
\end{itemize}


\subsection{Legal Types}

The given grammar of \gls{skill} already ensures that intuitive usage of the language will result in legal type declarations. The remaining aspects of illegal type declarations boil down to ill-formed usage of type and field names and can be summarized as:
\begin{itemize}
 \item Field names inside a type declaration must be unique inside the type and all its super types\footnote{The super type restriction may in fact be dropped?}.
 
 \item The subtype relation is a partial order\footnote{In fact it forms a forest.} and does not contain unknown types.

 \item For all fields f of dependent array type\footnote{E.g. a field \texttt{t[size] f} requires another field of integer type in the same declaration -- e.g. \texttt{i8 size}}, the size of the array has to denote a field of integer type in the very same declaration. The order of declaration is irrelevant.
 
 \item Any base type has to be known, i.e. it is either a ground type or it is a user type defined in any document transitively reachable over include commands.
\end{itemize}


\subsection{Type Order}

Let $<_l$ be the lexical order. We define a partial order $\leq_t$ on $\mathcal{T}$ as follows:
\begin{itemize}
 \item $\forall t \in \mathcal{B}, s \in \mathcal{T}\setminus\mathcal{B}. t \leq_t s$
 \item $\forall t \in \mathcal{C}, s \in \mathcal{U}. t \leq_t s$
 \item $\forall s,t \in \mathcal{U}. t \leq_t s \leftarrow s <: t $\footnote{This is \textit{super types first}.}
 \item $\forall s,t \in \mathcal{U}. t \leq_t s = t \leq_l s\leftarrow \exists S \in \mathcal{U} \cup \{\bot\}. t <: S \wedge s <: S $\footnote{Types with the same or no supertype are order lexically.}
\end{itemize}

The informal short description is, first ground types, then compound types and user types at the end, where the forest of user types maintains its structure but is order using the lexical order of type names.

Notice, that this order corresponds to an left to right order in the types overview picture.

The missing order of compound types is left away intentionally, because it allows for the exchange of some type definition after publishing a format, e.g. \verb/t[] f/ can be exchanged with \verb/list<t> f/.


\subsection{Strings}

Strings are conceptually a variable length sequence of utf8-encoded unicode characters. The in memory representation will try to make use of language features such as java.lang.String or std::u16string. The serialization is described in section \ref{???}. If a language demands 0-termination in strings, the language binding will ensure this.

Strings should not contain 0 characters, because this may cause problems with languages such as C.


\subsection{Compound Types}

The language offers several compound types. Sets, Lists and auto sized Arrays, i.e. arrays without an explicit size, are basically views onto the same kind of serialized data, i.e. they are a length encoded list of elements of the supplied base type. Arrays are expected to have a constant size, i.e. they are not guaranteed to be resizable. Sets are not allowed to contain the same element twice.
All ADTs will be mapped to their closest representation in the target language, while preserving these properties.
Maps are viewed as a representation of serializable partial functions. Therefore they can contain other map types as their second type argument, which is basically an instance of currying.

\subsection{NULL Pointer}

The null pointer is serialized using the index 0. Conceptually, null pointers of different types are different. In fact if an annotation is a null pointer, it still has a type. However, this detail should not be observable in most languages.


\subsection{Examples}

This section will present some examples of ill-formed type declarations and brief explanations.

\begin{lstlisting}[label=stringExample,caption=Legal Super Types,language=skill]
EncodedString : string {
  string encoding;
}
\end{lstlisting}
Error: The built-in type ``string'' can not be sub classed.

\section{Type Annotations}

\subsection{Restrictions}
\label{restrictions}

Some invariants can be added to declarations and fields. These invariants can occur at the same place as comments, but can occur in any number. Invariants start with an \textsc{@} followed by a predicate. Each predicate has to supply a default argument \texttt{\%}, such that using only default arguments would not imply a restriction.
If multiple predicates are annotated, the conjunction of them forms the invariant.
The set of legal predicates is explained below.

If predicates, which are not directly applicable for compound types are used on compound types, they expand to the contents of the compound types, if applicable. Otherwise the usage of the predicate is illegal.

\todo{hier muss man zwischen serialisierbaren und nicht serialisierbaren restrictions unterscheiden; serialisierbar sind alle restrictions, die auch auswirkungen auf die potentiell gespeicherten daten haben, wie range und nonnull}


\subsection*{Range}
Range restrictions are used to restrict integers and floats. Note that this will change the default value of the argument field to \textit{min} if $0 \notin [min,max]$.

Applies to fields: Integer, Float.

Signature: \verb/range(min, max)/: $\alpha \times \alpha → bool$

Defaults: the smallest/largest value of the argument type.

\begin{lstlisting}[label=rangeExample,caption=Examples,language=skill]
natural {
  @range(0,%)
  v64 data;
}
positive {
  @range(1,%)
  v64 data;
}
nonNegativeDouble {
  @range(0,%)
  f64 data;
}
\end{lstlisting}

\subsection*{NonNull}
Declares that an indexed field may not be null. Note that this will take away the default value and can therefore cause compatibility problems.

Applies to Field: Any indexed Type.

Signature: \verb/nonnull()/

Defaults: none.

\begin{lstlisting}[label=nonnullExample,caption=Examples,language=skill]
Node {
  @nonnull Node[] edges;
}
\end{lstlisting}


\subsection*{Unique}
Objects stored in a storage pool have to be distinct in their serialized form, i.e. for each pair of objects, there has to be at least one field, with a different value.
Because the combination of unique with sub-typing has counter intuitive properties, we decided that using the unique restriction together with a type that has sub- or super-types is considered an error.

Applies to Declarations.

Signature: \verb/unique()/

Defaults: none.

\begin{lstlisting}[label=uniqueExample,caption=Examples,language=skill]
@unique Operator {
  string name;
}
@unique Term {
  Operator operator;
  Term[] arguments;
}
\end{lstlisting}


\subsection*{Singleton}
There is at most one instance of the declaration.

Applies to Declarations.

Signature: \verb/singleton()/

Defaults: none.

\begin{lstlisting}[label=singletonExample,caption=Examples,language=skill]
@singleton System { ... }
@singleton Data{
  /** Note: if data would not be a singleton itself, it is likely to violate the singleton property */
  System foo;
}
\end{lstlisting}


% \subsection*{Integer Renaming}
% In some languages it is possible to define subsets of integers by renameing the integer type. The type declaration must contain exactly one field of some integer type. The resulting type will be treated as integer type. Language support is encouraged to make use of language specific features such as typedefs. In Ada, this restriction should be processed together with the range restriction.


\subsection*{Ascription}
A language specific type can be ascribed to a field. The type has to be compatible to the fields actual type, because the ascription will not change the ABI in any way. The first argument is the language name. The second type is generator dependent, but should be related to types as they occur in local variable or field declaration in the respective language.

Although this kind of restriction puts a heavy burden on the language generator and decreases readability a lot, it can be used to increase the usability of the generated interface a lot, because language features such es enums in Java or unions and bit fields in C++ can be used.

Applies to fields.

Signature: \verb/as(language, type)/: $\texttt{string} \times \texttt{string} → \{\}$

Defaults: not allowed.

\begin{lstlisting}[label=ascriptionExample,caption=Examples,language=skill]
System {
  /**
  The language binding makes use of an enumeration, which is supplied with the generated code.
  
  The C++ interface will use the different type using C-Casts to convert between the two types (which is completely fine if the enum uses char as a base type).
  
  The Java interface will assume the stored integer to be the ordinal of the enum SystemState.
  */
  @as("C++", "ccast SystemState")
  @as("Java", "enum SystemState")
  i8 state
}
\end{lstlisting}


\subsubsection*{Constant Length Pointer}
The pointer is serialized using i64 instead of v64. Can be used on regular references and annotations. This restriction makes only sense if the generated supports lazy reading of partial storage pool and if the files that have to be dealt with would not fit into the main memory of the target machine. Using this restriction will most certainly increase the file size.

This restriction is serializable and thus, does not affect compatibility in any way.

Applies to fields.

Signature: constantLengthPointer()\footnote{The length of the name is intended.}

Defaults: none.

\begin{lstlisting}[label=constantLengthPointerExample,caption=Examples,language=skill]
/* stored points to information may exceed the available main memory, thus we have to access it directly from disk */
PointsToTargets {
  @constantLengthPointer
  Context context;
  @constantLengthPointer
  HeapObject object;
  @constantLengthPointer
  PointsToSet targets;
}
\end{lstlisting}




\subsection{Hints}
\label{hints}

Hints are annotations that start with a single \verb/!/ and are followed by a hint name. Hints are used to control the behavior of the generated language binding and do not have an impact on the semantics of the stored data. Therefore they will not be stored in the reflection pool.

\subsection*{Access}
Try to use a data structure that provides fast (random) access. E.g. an array list.

\subsection*{Modification}
Try to use a data structure that provides fast (random) modification. E.g. a linked list\footnote{Which has faster insert/delete operations than an array list.}.

\subsection*{Unique}
\label{hints:unique}
Serialization shall unify objects with exactly the same serialized form. In combination with the @unique restriction, there shall at most be an error reported on deserialization.


\subsection*{Pure}
\label{hints:pure}
The generated deserialization shall fork objects upon modification. An example of this behavior is the string pool. An equivalent, in terms of API observable behavior, would look as follows:
\begin{lstlisting}[label=prueExample,caption=User Strings,language=skill]
!pure
!unique
UserString {
  i8[] utf8Chars;
}
\end{lstlisting}



\subsection*{Distributed}
Use a static map instead of fields to represent fields of definitions in memory. This is usually an optimization if a definition has a lot of fields, but most use cases require only a small subset of them. Because hints do not modify the binary compatibility, some clients are likely to define the fields to be distributed or even lazy. Note that this will increase both the memory footprint and the access time for the given field and will only be a benefit for memory-cache locality reasons. The internal representation will change from \texttt{f.a}, i.e. a regular field, to \texttt{pool.a[f]}, i.e a map in the storage pool which holds the field data for each instance. Note that the presence of distributed, lazy or ignored fields will require objects to carry a pointer to their storage pool, which may eliminate the cache savings completely.

\subsection*{Lazy}
Deserialize the fields data only if it is actually used. Lazy implies distributed.

\subsection*{ReadOnly}
The generated code is unable to modify the respective field or instances of the respective type. This options is provided to provide a consistent API while preventing from logical errors, such as modifying data from a previous stage of computation.

\subsection*{Ignore}
The generated code is unable to access the respective field or any field of the type of the target declaration. This will lead to errors, if it is tried nonetheless. This option is provided to provide a consistent API for a combined file format, but restrict usage of certain fields, which should be transparent to the current stage of computation.

\section{Serialization}
\label{serialization}

This section is about representing objects as a sequence of bytes. We will call this sequence \textit{stream}, its formal Type will be named $S$, the current stream will be named $s$. We will assume that there is an implicit conversion between fixed sized integers\footnote{As well as between fixed sized floating point numbers, because we define them to be IEEE-754 encoded 32-/64-bit sequences.} and streams. We also make use of a stream concatenation operator $\circ : S \times S → S$.

This section assumes, that all objects about to be serialized are already known. It further assumes, that their types and thus the values of the functions (i.e. baseTypeName, typeName, index, $\den{\_}$) explained below can be easily computed.

The serialization function $\den{\_}_\tau : \tau \times \mathcal{T} → S$ will be written simply as \den{\_} if $\tau$ is clear form the context.


\subsection{Steps of the Serialization Process}

In general it is assumed that the serialization process is split into the following steps:
\begin{enumerate}
 \item All objects to be serialized are collected. This is usually done using the transitive closure of an initial set.
 
 \item The items are organized into their storage pools, i.e. the index function is calculated.\todo{das ist nur die halbe wahrheit, weil man hier auch noch updates(im sinne von dynamischer logik:)) für modifikationen machen muss}
 
 \item The output stream is created as described below.
\end{enumerate}

\subsection{General File Layout}

The file layout is optimized for lazy loading of stored data. It does also support type-safe and consistent treatment of unknown data structures. In order to achieve this, we have to store the type system used by the file together with the stored data. The type system itself is using strings for its representation, thus we have motivated the following layout.

\begin{verbatim}
utf8[][] stringPool

while(!EOF){
  string typeName
  string superTypeName
  option(v64 basePoolStartIndex; iff has superType)
  v64 sizeCount
  [[restrictions]]
  v64 fieldCount
  foreach f in fields {
    [[f.restrictions]]
    [[f.type]]
    string f.name
    @as(f.T[sizeCount])
    i8[] f.elements
  }
}
\end{verbatim}


\subsection{Storage Pools}

This section contains the serialization function for an individual storage pool. We assume that storage pools are not empty. If an empty storage pool would be written to disk, it is simply skipped.\footnote{This has the side effect, that only type information of instantiated types are present.}

Writing objects of a pool requires the following functions: $baseTypeName: \mathcal{U} → S$, $typeName: \mathcal{U} → S$ and $index: \mathcal{U}\cup\{\textbf{string}\} → S$.

The basic idea behind the serialization format is to store the data grouped by type into storage pools. If objects are referred to from other objects, those references are given as an integer, which is interpreted as index into the respective storage pool. The NULL pointer is represented by the index 0.

Each pool keeps a start index, which allows for the reconstruction of the complete object. A short example shall illustrate the basic concept. It contains five types A,B,C,D and N. Each has a single field of type $\tau$ which is used to simplify the representation. The type information for the objects in the base type pool can be inferred from the data stored in the pools using the links between the base type pool and the subtype pools (The \gls{baseType} start index (BPSI) field of pools with a super type -- shown as arrows). For the sake of readability, the name, size and count fields are omitted in the picture.

\begin{figure}[h]

%NOTE: if this picture is used in a paper, colors have to be replaced by stippled lines
\begin{tikzpicture}
%labels
\node[draw=none] at (-1,0.25) {$S$:};
\node[draw=none] at (-1,-0.25) {$\mathcal{T}$:};
\node[draw=none] at (-0.7,-1) {$index$:};
\node[draw=none] at (-0.85,-0.65) {\small{BPSI}:};
\node[draw=none] at (-0.9,2) {Def.:};

%definitions
\node[red,draw=none] at (2.5,2) {A \{ $\tau a$; \} };
\node[green,draw=none] at (1.5,1.5) {B : {\color{red} A } \{ $\tau b$; \} };
\node[blue,draw=none] at (1,1) {D : {\color{green} B } \{ $\tau d$; \} };
\node[black,draw=none] at (4,1.5) {C : {\color{red} A } \{ $\tau c$; \} };

\node[orange,draw=none] at (7.5,2) {N \{ $\tau n$; \} };

%content
 \node[color=black!60,draw=none] at (-0.35,0.25) {$\cdots$};
 
 \node[red,draw=none] at (0.25,0.25) {${a}$};
 \node[red,draw=none] at (0.75,0.25) {${a}$};
 \node[red,draw=none] at (1.25,0.25) {${a}$};
 \node[red,draw=none] at (1.75,0.25) {${a}$};
 \node[red,draw=none] at (2.25,0.25) {${a}$};
 \node[red,draw=none] at (2.75,0.25) {${a}$};
 
 \node[green,draw=none] at (3.45,0.25) {$b$};
 \node[green,draw=none] at (3.95,0.25) {${b}$};
 \node[green,draw=none] at (4.45,0.25) {${b}$};
 \node[green,draw=none] at (4.95,0.25) {${b}$};
 
 \node[blue,draw=none] at (5.65,0.25) {${d}$};
 
 \node[black,draw=none] at (6.35,0.25) {${c}$};
 
 \node[orange,draw=none] at (6.9,0.25) {${n}$};
 \node[orange,draw=none] at (7.55,0.25) {$\cdots$};
 
%base pool index
 
 \node[red,draw=none] at (0.25,-1) {1};
 \node[red,draw=none] at (0.75,-1) {2};
 \node[red,draw=none] at (1.25,-1) {3};
 \node[red,draw=none] at (1.75,-1) {4};
 \node[red,draw=none] at (2.25,-1) {5};
 \node[red,draw=none] at (2.75,-1) {6};
 
 \node[green,draw=none] at (3.45,-1) {2};
 \node[green,draw=none] at (3.95,-1) {3};
 \node[green,draw=none] at (4.45,-1) {4};
 \node[green,draw=none] at (4.95,-1) {5};
 
 \node[blue,draw=none] at (5.65,-1) {5};
 
 \node[black,draw=none] at (6.35,-1) {6};
 
 \node[orange,draw=none] at (6.9,-1) {1};
 \node[orange,draw=none] at (7.55,-1) {$\cdots$};
 
%base pool types
 \node[red,draw=none] at (0.25,-0.25) {A};
 \node[green,draw=none] at (0.75,-0.25) {B};
 \node[green,draw=none] at (1.25,-0.25) {B};
 \node[green,draw=none] at (1.75,-0.25) {B};
 \node[blue,draw=none] at (2.25,-0.25) {D};
 
 \node[black,draw=none] at (2.75,-0.25) {C};
 
 \node[orange,draw=none] at (6.9,-0.25) {N};

%frames
 \draw[shift={(0.15,0)},thick,orange] (6.4999,0) grid [step=0.5] (7.4,0.5);

 \draw[shift={(0.6,0)},thick,black] (5.3,0) grid [step=0.5] (6,0.5);
 \draw[shift={(0.4,0)},thick,blue] (4.8,0) grid [step=0.5] (5.5,0.5);
 \draw[shift={(0.2,0)},thick,green] (2.8,0) grid [step=0.5] (5,0.5);
 \draw[thick,red] (0,0) grid [step=0.5] (3,0.5);
 
%base type indices
 \draw[thick,green,->] (3.1,0.1) -- (3.1,-0.5) -- (0.5,-0.5) -- (0.5,0);
 \draw[thick,blue,->] (5.3,0.1) -- (5.3,-0.6) -- (2,-0.6) -- (2,0);
 \draw[blue,->] (5.3,0.1) -- (5.3,-0.6) -- (4.7,-0.6) -- (4.7,0);
 \draw[thick,black,->] (6,0.1) -- (6,-0.7) -- (2.5,-0.7) -- (2.5,0);
\end{tikzpicture}
\caption{The serialization scheme used to store objects into pools.}
\end{figure}

The order in which pools are serialized is currently unrestricted.


\subsection{Pool Elements}

In this section, we want to describe the serialization of individual fields using the function $\den{\_}_\tau$. The serialization of an objects takes places by serializing all its fields into the stream. In this section, we assume that the three functions defined in the last section are implicitly converted to streams using the v64 encoding. We assume further, that compound types provide a function $size: \mathcal{T} → \mathcal{I}$, which returns the number of elements stored in a given field.
Let $f$ be a field of type $t$, then $\den{f}$ is defined as\footnote{We will use C-Style hexadecimal integer literals for integers in streams.}
\begin{itemize}
 %pooled objects
 \item $\forall t \in \mathcal{U}\cup\{\textbf{string}\}. \den{f}_t = \left\{ 
   \begin{array}{l l}
     \texttt{0x00}, & f = \texttt{NULL}\\
     index(f) & else
   \end{array} \right.$
 
 %annotation -> * (v64 baseTypeName!!, v64index)
 \item $\den{f}_{\textbf{annotation}} = \left\{ 
   \begin{array}{l l}
     \texttt{0x00 0x00}, & f = \texttt{NULL}\\
     baseTypeName(f) \circ index(t) & else
   \end{array} \right.$
   \footnote{We do not want to use type IDs here, because we do not want to touch all annotation fields if we modify the type Pool.}
 
 %bool
 \item $\den{\top}_\textbf{bool} = \texttt{0xFF}$
 \item $\den{\bot}_\textbf{bool} = \texttt{0x00}$
 
 %fixed int
 \item $\forall t \in \mathcal{I}\setminus\{\textbf{v64}\}. \den{f}_t = f$
 
 %v64
 \item $\den{f}_\textbf{v64} = encode(f)$\footnote{With encode as defined in listing \ref{v64enc}.}
 
 %fixed float
 \item $\den{f}_\textbf{f32} = \den{f}_\textbf{f64} = f$\footnote{Assuming the float to be IEEE-754 encoded, which allows for an implicit bit-wise conversion to fixed sized integer.}
 
 %fixed and dependent arrays
 \item $\forall g \in \mathcal{B}, n \in \mathbb{N}^+. t = g\texttt{[}n\texttt{]} \implies \den{f} = \den{f_0}_g \circ \cdots \circ \den{f_{n-1}}_g$
 
 \item $\forall g \in \mathcal{B}, s \in \mathcal{I}, \texttt{s size}$\footnote{As stated above, size must be a field of the same declaration as f.} $. t = g\texttt{[size]} \wedge \texttt{size} > 0 \implies \den{f} = \den{f_0}_g \circ \cdots \circ \den{f_{\texttt{size}-1}}_g$\footnote{Note that this is the only case where the encoded field does not append anything to the stream.}
 
 %variable array, list and set
 \item $\forall g \in \mathcal{B}, n = size(f), t \in \{g\texttt{[]}, \texttt{set<}g\texttt{>}, \texttt{list<}g\texttt{>}\}. \den{f} = \den{n}_{v64} \circ \den{f_0}_g \circ \cdots \circ \den{f_{n-1}}_g$
 
 %map
 \item Maps are serialized from left to right by serializing the keyset and amending each key with the map structure which it points to. In case of Maps with two types, this is equal to a list of key value tuples.
 A field of type \verb/map<T,U,V>/ is serialized using a schema $ \den{size(f)} \circ \den{f.t_1} \circ \den{size(f[t_1])} \circ \den{f[t_1].u_1} \circ \den{f[t_1][u_1]} \circ \den{f[t_1].u_2} \circ \cdots \circ \den{size(f[t_2])} \circ \cdots \circ \den{f[t_n][u_m]}$. Note that we treat maps like map<T, map<U,V>>.
 
 %restrictions
 \item $\den{RESTRICTION} = \left\{ 
   \begin{array}{l l}
     \emptyset, & id = \bot\\
     \den{id}_{v64}\den{arg_1}_{string}\circ \cdots \circ \den{arg_n}_{string} & else
   \end{array} \right.$
 
 %types
 \item $\den{t}_{type} = \left\{ 
   \begin{array}{l l}
   \den{id}_{i8} \circ \den{val}_{t} & id \in [0,4] \\
   \den{id}_{i8} & id \in [5,14] \\
   15 \circ \den{i}_{v64} \circ \den{T} & t = T[i]\\
   16 \circ \den{f.nameIndex}_{v64} \circ \den{T} & t = T[f]\\
   17 \circ \den{T} & t = T[]\\
   18 \circ \den{T} & t = list<T>\\
   19 \circ \den{T} & t = set<T>\\
   20\circ \den{n}_{v64} \circ \den{T_1} \circ \cdots \circ \den{T_n} & t = map<T_i,\ldots,T_n>\\
   \den{21+reflectionPoolIndex(t)}_{v64} & t \in \mathcal{U}\\
   \end{array} \right.$
\end{itemize}

\subsection{Endianness}

Files are stored in a little endian format, which is the default for common architectures.

If a client is running on a big endian machine, the endianness has to be corrected, both when reading and writing files. This can be done by changing the implementation of $\den{\_}_{i*}$- and $\den{\_}_{f*}$-translations.



\section{Deserialization}

Deserialization is mostly straight forward.

The general strategy is:
\begin{itemize}
 \item the string pool is deserialized into an array
 \item the reflection pool is deserialized using the strings array
 \item the structure of storage pools is read, pools are created and chunks of field data are copied into memory
 \item required fields are parsed using the information from the reflection pool
\end{itemize}

\subsection*{Date Example}

Let $d$ be the deserialization function -- basically the inverse function of \den{\_}.

$d(01 04 64 61  74 65 01 00  02 00 01 00  0B 01 0A 01   FF FF FF FF  FF FF FF FF  FF)$\\
$=d(01) d(04 64 61  74 65 01 00  02 00 01 00  0B 01 0A 01   FF FF FF FF  FF FF FF FF  FF)$\\
$=d(01) d(04) d(64 61  74 65 01 00  02 00 01 00  0B 01 0A 01   FF FF FF FF  FF FF FF FF  FF)$
$=d(01) d(04) d(64 61  74 65) d(01 00  02 00 01 00  0B 01 0A 01   FF FF FF FF  FF FF FF FF  FF)$
$=string[1:"date"] d(01 00  02 00 01 00  0B 01 0A 01   FF FF FF FF  FF FF FF FF  FF)$
$=string[1:"date"] d(01) d(00) d(02) d(00) d(01) d(00  0B 01 0A 01   FF FF FF FF  FF FF FF FF  FF)$
$=string[1:"date"] date[T:date[\_] 1:\_ 2:\_ d(00  0B 01 0A 01   FF FF FF FF  FF FF FF FF  FF)$
$=string[1:"date"] date[T:date[d(00  0B 01)] 1:\_ 2:\_ d(0A 01   FF FF FF FF  FF FF FF FF  FF)$
$=string[1:"date"] date[T:date[v64 date] 1:\_ 2:\_ d(0A 01   FF FF FF FF  FF FF FF FF  FF)$
$=string[1:"date"] date[T:date[v64 date] 1:\_ 2:\_ d(01   FF FF FF FF  FF FF FF FF  FF)]$
$=string[1:"date"] date[T:date[v64 date] 1:[1] 2:[-1]]$
\section{API}

The generated \gls{api} has to be designed in a fashion that integrates nicely with the languages programming paradigms. E.g. in Java it would be most useful to create a state object, which holds state of a bunch of serializable data and provides iterators over existing objects, as well as factory methods and methods to remove objects form the state object. The serialized types can be represented by interfaces providing getters, setters, using hidden implementations, only known to the state object.

talk about the generated API and its features, like iterators, factories, access to singletons and stuff.

\subsection{Examples}

Nice example in C++:
\begin{lstlisting}[label=cppExample,caption=C++ Examples,language=C++]
#include <stdint.h>
#include <string>
[...some other bouilerplate includes...]
struct SLoc {
  uint16_t line;
  uint16_t column;
  std::string* path;
};
struct Block {
  std::string* tag;
  SLoc* begin;
  SLoc* end;
  std::string* image;
};
struct IfBlock : public Block {
  Block thenBlock;
};
struct ITEBlock : public IfBlock {
  Block elseBlock;
};
[...
  plus some boilerplate code for visitors, iostreams etc.
...]
\end{lstlisting}

\begin{lstlisting}[label=javaExample,caption=Java Examples,language=Java]
class SLoc {
  public short line;
  public short column;
  public String path;
}
class Block {
  final public String tag() {
    return this.getClass().getName();
  }
  public SLoc begin;
  public SLoc end;
  public String image:
}
class IfBlock extends Block {
  public Block thenBlock;
}
class ITEBlock extends IfBlock {
  public Block elseBlock;
}
[...some read and write code, plus some visitors...]
\end{lstlisting}


\begin{lstlisting}[label=latexExample,caption=LaTeX Examples,language={[LaTeX]TeX}]
$(line, column, path) \in SLoc
  \subseteq \mathbb{Z} \times \mathbb{Z} \times string$

$(begin, end, image) \in Block
  \subseteq SLoc \times SLoc \times string$

$(super, thenBlock) \in IfBlock
  \subseteq Block \times Block$

$(super, elseBlock) \in ITEBlock
  \subseteq IfBlock \times Block$
\end{lstlisting}
Which looks like:

$(line, column, path) \in SLoc \subseteq \mathbb{Z} \times \mathbb{Z} \times string$

$(begin, end, image) \in Block \subseteq SLoc \times SLoc \times string$

$(super, thenBlock) \in IfBlock \subseteq Block \times Block$

$(super, elseBlock) \in ITEBlock \subseteq IfBlock \times Block$

Note: The incentive of the \LaTeX-output is to provide a mechanism for users to formalize their file format using mechanisms, that are or can not be available as a specification language. E.g. the sentence ``The path of a SLoc points to a valid file on the file system and the line and column form a valid location inside that file.'' can not be verified in a static manner. This is because the correctness of the property depends not only on the content to be verified, but on the verifying environment as well.



\section{Future Work}

XML output mit XML Schema.

Das neue Serialisierungsschema erlaubt es einen Viewer zu bauen, der Definition+Datei anzeigen kann. (Die future work ist hier der viewer)

Integration der Definition in die Serialisierte Form, damit man die Daten generisch prüfen und anzeigen kann. Hier braucht man noch ein gutes encoding, weil man sonst zu viel platz verbraucht.

Abuse annotations for type-safe unions. The type system does not allow for unrestricted unions or intersection types. The former violate serialization invariants, the latter would either have no instances or be equal to an already existing (super) type.

State somewhere, that a major advantage over XML is, that one is not required to link against a overly general implementation, which is nice if one is only interested in a very specific format.

A notion of \textit{first class strings} can be used to separate the string pool into one pool at the beginning of the file which contains all the strings, that have to be used in order to understand the contents of the file and a second part of the pool, which can be skipped. This should give significant performance improvements if files with lots of unused strings are processed.

True comments with \verb/# ... \n/?

Alternativ kann man die reflection data aus dem ReflectionPool als Eingabesprache für einen Generator benutzen.

Fun fact: Garbage Collection for serializable objects comes for free, if objects are always held in storage pools.

Add generic/template import statements, which allow to import files together with a substitution. That way one can create more complex ADTs such as B-Trees. This feature is not a priority and only useful for large files and projects and requires ascription to yield the desired effect automatically.

Add \textit{add} and \textit{subtract} declarations to the file format, starting with \texttt{++} or \textit{--}, which allow taking away or adding fields to types. If this is done, a usability evaluation is necessary!

\newpage
\todos

\part{Appendix}
\renewcommand\thesection{\Alph{section}}
\setcounter{section}{0}

\section{Variable Length Coding}

Size and Length information is stored as variable length coded 64 bit unsigned integers (aka C's \texttt{uint64\_t}). The basic idea is to use up to 9 bytes, where any byte starts with a 1 iff there is a consecutive byte. This leaves a payload of 7 bit for the first 8 bytes and 8 bits of payload for the ninth byte. This is very similar to the famous utf8 encoding and is motivated, as it is the case with utf8, by the assumption, that smaller numbers are a lot more likely. It has the nice property, that there are virtually no numerical size limitations.
The following small C++ functions will illustrate the algorithm:
\begin{lstlisting}[label=v64enc,caption=Variable Length Encoding,language=C++]
uint8_t* encode(uint64_t v){
  // calculate effective size
  int size = 0;
  {
    auto q = v;
    while(q){
      q >>= 7;
      size++;
    }
  }
  if(!size){
    auto rval = new uint8_t[1];
    rval[0]=0;
    return rval;
  }else if(10==size)
    size = 9;

  // split
  auto rval = new uint8_t[size];
  int count=0;
  for(;count<8&&count<size-1;count++){
    rval[count] = v >> (7*count);
    rval[count] |= 0x80;
  }
  rval[count] = v >> (7*count);
  return rval;
}
\end{lstlisting}
\begin{lstlisting}[label=v64dec,caption=Variable Length Decoding,language=C++]
uint64_t decode(uint8_t* p){
  int count = 0;
  uint64_t rval = 0;
  register uint64_t r;
  for(;count<8 && (*p)&0x80; count++, p++){
    r = p[0];
    rval |= (r&0x7f)<<(7*count);
  }
  r = p[0];
  rval |= (8==count?r:(r&0x7f))<<(7*count);
  return rval;
}
\end{lstlisting}
\section{Error Reporting}

This section describes some errors regarding ill-formatted files, which must be detected and reported. The order is based on the expected order of checking for the described error. The described errors are expected to be the result of file corruption, format change or bugs in a language binding.

\subsection*{Deserialization}
\begin{itemize}
\item If \texttt{EOF} is encountered unexpectedly, an error must be reported before producing any observable result.

\item If an index into a pool is invalid\footnote{because it is larger then the last string in the pool}, an error must be reported.

\item If the deserialization of a storage pool does not consume exactly the \texttt{sizeBytes} in its header, an error must be reported. Note: This is a strong indicator for a format change.

\item If the serialized type order of storage pools does not match the expected type order, an error must be reported.

\item If the serialized type information contains cycles, an error must be reported, which contains at least all type names in the detected cycle and the base type, if one can be determined.

\item If a storage pools contains elements which, based on their location in the base pool, should be subtypes of some kind, but have no respective sub type storage pool, an error must be reported with at least, the base type name, the most exact known type name and the adjacent base type names.

\item All known constant fields have to be checked before producing any observable result. If some constant value differs from the expected value, an error must be reported, which contains at least the type, the field type and name, the basePoolIndex, the index inside the types pool, the expected value and the actual value.

\item If a serialized value violates a restriction or the invariant of a type,\footnote{Including sets containing multiple similar elements.} an error must be reported as soon as this fact can be observed. It is explicitly not required to check all serialized data for this property.
\end{itemize}



\section{Reserved Words}
\label{app:keywords}

This section contains a table of words which must not be used as field names, because they are keywords in some languages. The usage of skill keywords will result in a direct error, whereas the usage of a word listed below will result in a warning, because the identifier will be escaped in the target language binding.


\textbf{
\begin{tabular}{ccccc}
if &then& else &begin &end\\
struct & class & public & protected &private \\
⇒ & ...
\end{tabular}
}


\section{Core Language}
The core language is a subset of the full language which must be supported by any generator, which is called skill core language generator. Features included in the core language are:
\begin{itemize}
 \item Integer types \texttt{i8} to \texttt{i64} and \texttt{v64}
 \item \texttt{string}, \texttt{bool} and \texttt{annotation}
 \item Compound types
 \item User Types with sub-typing
 \item \texttt{const} and \texttt{auto} fields.
 \item Reflection.
\end{itemize}

Thus the remaining parts required for full skill support are:
\begin{itemize}
 \item Floats
 \item Restrictions
 \item Hints
 \item Language dependent treatment of comments, e.g. integration into doxygen or javadoc.\footnote{This may even require a language extension providing tags inside comments which are translated into tags of the respective documentation framework.}
 \item Name mangling to allow for usage of language keywords or illegal characters (unicode) in specification files, without making a language binding impossible.
\end{itemize}



\section{Numerical Limits}

In order to keep serialized data platform independent, one has to respect the numerical limits of the various target platforms. For instance, the Java Virtual Machine will not allow arrays with a size larger then $2^31$ minus some elements. Therefore we establish the following rule:

(De-)serialization of a file with an array of more then $2^30$ elements or a type with more then $2^30$ instances may fail due to numerical limits of the target platform.

\section{Numerical Constants}
\label{app:constants}

This section will list the translation of type IDs(as required in section \ref{serialization:elements}) and restriction IDs (see section \ref{restrictions} and \ref{serialization:elements}). Restrictions with undefined IDs will not be serialized.

\begin{table}
\begin{subtable}[t]{5cm}
\centering
\begin{tabular}{l|l}
Type Name & Value \\\hline
const i8 	& 0 \\
const i16 	& 1 \\
const i32 	& 2 \\
const i64 	& 3 \\
const v64 	& 4 \\
annotation	& 5 \\
bool		& 6 \\
i8 		& 7 \\
i16 		& 8 \\
i32 		& 9 \\
i64 		& 10 \\
v64 		& 11 \\
f32 		& 12 \\
f64 		& 13 \\
string		& 14 \\
T[i] 		& 15 \\
T[f] 		& 16 \\
T[] 		& 17 \\
list<T> 	& 18 \\
set<T>	 	& 19 \\
map<T$_1$, \ldots, T$_n$> & 20 \\
T 		& 21 + $index_T$ \\
\end{tabular}
\caption{Type IDs}
\end{subtable}
~
\begin{subtable}[t]{5cm}
\centering
\begin{tabular}{l|l}
Restriction Name & Value \\\hline
range 	& 0 \\
nonnull 	& 1 \\
unique 	& 2 \\
singleton 	& 3 \\
\ldots 	& \ldots \\
\end{tabular}
\caption{Restriction IDs}
\end{subtable}
\end{table}



\newpage
%%% TODO REMOVE NEXT LINE
%\glsaddall
\printglossaries

\bibliographystyle{alpha}
\bibliography{skill}

\end{document}
